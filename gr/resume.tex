%!TEX TS-program = xelatex
%!TEX encoding = UTF-8 Unicode

\documentclass[11pt, a4paper]{article}

% LAYOUT
%--------------------------------
% Margins
\usepackage{geometry}
\geometry{a4paper, left=35mm, right=35mm, top=25mm, bottom=25mm}

% Do not indent paragraphs
\setlength\parindent{0in}

% Enable multicolumns
\usepackage{multicol}
\setlength{\columnsep}{-3.5cm}

% Uncomment to suppress page numbers
% \pagenumbering{gobble}

% LANGUAGE
%--------------------------------
\usepackage{polyglossia}
\setmainlanguage[variant=mono]{greek}
\PolyglossiaSetup{greek}{indentfirst=false}
\setotherlanguage[variant=us]{english}
% add hyp rules or replace compound words with \hyp{}
\usepackage{hyphenat}
\hyphenation{αν-θρώ-που-υπο-λο-γι-στή}

% TYPOGRAPHY
%--------------------------------
\usepackage{fontspec}
\usepackage{xunicode}
\usepackage{xltxtra}
% converts LaTeX specials (quotes, dashes etc.) to Unicode
\defaultfontfeatures{Mapping=tex-text}
\setromanfont [Ligatures={Common}, Numbers={OldStyle}]{GFS Didot}
% Cool ampersand
\newcommand{\amper}{{\fontspec[Scale=.95]{GFS
Didot}\selectfont\itshape\&}}
% nice icons
\usepackage{fontawesome5}
\usepackage{academicons}
% reduce widows and orphans
\usepackage[all]{nowidow}
\setlength{\emergencystretch}{3em}  % prevent overfull lines
\providecommand{\tightlist}{%
  \setlength{\itemsep}{0pt}\setlength{\parskip}{0pt}}

\usepackage[protrusion=true,final]{microtype}
\usepackage{hyperref}


% MARGIN NOTES
%--------------------------------
\usepackage{marginnote}
\newcommand{\note}[1]{\marginnote{\scriptsize #1}}
\renewcommand*{\raggedleftmarginnote}{}
\setlength{\marginparsep}{7pt}
\reversemarginpar

% HEADINGS
%--------------------------------
\usepackage{sectsty}
\usepackage[normalem]{ulem}
\sectionfont{\rmfamily\mdseries}
\subsectionfont{\rmfamily\mdseries\scshape\normalsize}
\subsubsectionfont{\rmfamily\bfseries\upshape\normalsize}
\setcounter{secnumdepth}{0}

% PDF SETUP
%--------------------------------
\usepackage{hyperref}
\hypersetup
{
  pdfauthor={Κωνσταντίνος Χωριανόπουλος},
  pdfsubject={Κωνσταντίνος Χωριανόπουλος CV},
  pdftitle={Κωνσταντίνος Χωριανόπουλος CV},
  colorlinks, breaklinks, xetex, bookmarks,
  filecolor=black,
  urlcolor=[rgb]{0.117,0.682,0.858},
  linkcolor=[rgb]{0.117,0.682,0.858},
  linkcolor=[rgb]{0.117,0.682,0.858},
  citecolor=[rgb]{0.117,0.682,0.858}
}

%
%bibliography
\usepackage[backend=biber,style=numeric,firstinits=true,sorting=ynt,maxnames=5,minnames=3,hyperref=false,isbn=false,doi=false,url=false,safeinputenc=true,language=english]{biblatex}
\usepackage{csquotes}
\usepackage{hyperref}
\DeclareNameAlias{default}{last-first}

%J1, J2, etc
\makeatletter
\define@key{blx@bib2}{prefixnumbers}{%
  \def\blx@prefixnumbers{#1}%
  \iftoggle{blx@defernumbers}
    {}
    {\iftoggle{blx@labelnumber}
       {\blx@warning{%
          Option 'prefixnumbers' requires global\MessageBreak
          'defernumbers=true'}}
       {}}}
\makeatother

% table of contents without dots
\usepackage{tocloft}
\renewcommand{\cftdot}{}

% bibliography sections
\makeatletter
\newcommand{\bibsection}[3]{
\subsection{#1}
\begin{refsection}[#2]
  \nocite{*}
  \begin{refcontext}[prefixnumbers=#3]
    \printbibliography[heading=none]
  \end{refcontext}
\end{refsection}
}
\makeatother

% titlepages style
\newlength{\drop}% for my convenience

%titleS
\makeatletter
\newcommand*{\titleS}{\begingroup% Scripts, T&H p 151
\drop = 0.1\textheight
\centering
\vspace*{\drop}
{\Huge \@title}\\[\baselineskip]
{\large\itshape \@author}\\[\baselineskip]
\vfill
\rule{0.4\textwidth}{0.4pt}\\[\baselineskip]
{\large\itshape \@date}\par
\vspace*{\drop}
\endgroup}
\makeatother

%titleGM
\makeatletter
\newcommand*{\titleGM}{\begingroup% Gentle Madness
\drop = 0.1\textheight
  \hbox{%
  \hspace*{0.1\textwidth}%
  \rule{1pt}{\textheight}
  \hspace*{0.05\textwidth}%
  \parbox[b]{0.75\textwidth}{
  \vbox{%
    \vspace{\drop}
    {\noindent\Huge\bfseries Κωνσταντίνος\\[0.5\baselineskip]
               Χωριανόπουλος}\\[2\baselineskip]
    {\Large\itshape Αναπληρωτής Καθηγητής\\[0.5\baselineskip]
                    Τεχνολογίες Λογισμικού\\[0.5\baselineskip]
                    Ιόνιο Πανεπιστήμιο}\\[4\baselineskip]
    \textsc{\LARGE \@title}\par
    \vspace{0.5\textheight}
    {\noindent \@date}\\[\baselineskip]
    }% end of vbox
    }% end of parbox
  }% end of hbox
\endgroup}
\makeatother

\title{Αναλυτικό Βιογραφικό Σημείωμα}
\author{Κωνσταντίνος Χωριανόπουλος, Επίκουρος Καθηγητής (Τεχνολογίες Λογισμικού)}

% DOCUMENT
%--------------------------------
\begin{document}

\pagestyle{empty}
\titleGM
\clearpage

\tableofcontents
\clearpage

\pagestyle{plain}

\begin{greek}

{\LARGE Κωνσταντίνος Χωριανόπουλος}\\[.2cm]
\textit{Αναπληρωτής Καθηγητής (Τεχνολογίες Λογισμικού)}\\[.2cm]

\begin{multicols}{2}

{\scriptsize
Ιόνιο Πανεπιστήμιο\\
Τμήμα Πληροφορικής\\
Πλατεία Τσιριγώτη 7\\
49100 Κέρκυρα\\
Ελλάδα\\
}

\columnbreak

{\scriptsize
\faStyle{regular}
\faHome \hspace{0.1cm} \href{http://users.ionio.gr/choko}{users.ionio.gr/choko}\\
\faEnvelope \hspace{0.1cm} \href{mailto:choko@ionio.gr}{choko@ionio.gr}\\
\aiOrcid \hspace{0.1cm} \href{http://orcid.org/0000-0002-5999-9387}{ORCID}\\

\vfill

\faPhone +30 26610 87707\\
}

\end{multicols}

\vspace{30pt}


\section{Ερευνητικά ενδιαφέροντα}
          \quad Διάδραση ανθρώπου και υπολογιστή
          \quad Τεχνολογίες λογισμικού
          \quad Προγραμματισμός πολυμέσων
          \quad Διάδραση με βίντεο
          \quad Κινητός και διάχυτος υπολογισμός
          \quad Κοινωνικά και συνεργατικά συστήματα
          \quad Επικοινωνία και μάθηση
    
% \vfill

\vspace{25pt}

\section{Εργασία}
\noindent
\note{2008---}\textsc{Ιόνιο Πανεπιστήμιο (ΙΠ)}\\
\emph{Αναπληρωτής Καθηγητής (2020--)\\
Επίκουρος Καθηγητής (2014--2020)\\
Λέκτορας (2009--2014)\\
Συμβασιούχος διδάσκοντας (2008--2009)}\\
Κέρκυρα, Ελλάδα\\[.2cm]
\note{2006---}\textsc{Ελληνικό Ανοικτό Πανεπιστήμιο (ΕΑΠ)}\\
\emph{Συνεργαζόμενο Εκπαιδευτικό Προσωπικό (ΣΕΠ)}\\
Αθήνα, Ελλάδα\\[.2cm]
\note{2006---2008}\textsc{Bauhaus-University of Weimar}\\
\emph{Μεταδιδακτορικός ερευνητής}\\
Βαϊμάρη, Γερμανία\\[.2cm]
\note{2006---2008}\textsc{Πανεπιστήμιο Αιγαίου}\\
\emph{Συμβασιούχος διδάσκοντας}\\
Σύρος, Ελλάδα\\[.2cm]
\note{2005--2006}\textsc{Ελληνικό Πολεμικό Ναυτικό}\\
\emph{Ναύτης}\\
Χανιά-Αθήνα, Ελλάδα\\[.2cm]
\note{2004--2005}\textsc{Imperial College London}\\
\emph{Μεταδιδακτορικός ερευνητής}\\
Λονδίνο, Αγγλία\\[.2cm]
\note{2000--2004}\textsc{Οικονομικό Πανεπιστήμιο Αθήνας}\\
\emph{Μεταπτυχιακός ερευνητής}\\
Αθήνα, Ελλάδα\\[.2cm]
\note{1997--1999}\textsc{Πολυτεχνείο Κρήτης}\\
\emph{Προπτυχιακός ερευνητής}\\
Χανιά, Ελλάδα\\[.2cm]

\section{Εκπαίδευση}
\noindent
\note{2004}\textsc{Διδακτορικό}\\
\emph{Virtual television channels: Conceptual model, user interface
design and affective usability evaluation, Επίβλεψη: Διομήδης
Σπινέλλης}\\
\emph{Διοικητική Επιστήμη και Τεχνολογία, Οικονομικό Πανεπιστήμιο
Αθήνας}.\\[.2cm]
\note{2001}\textsc{Μεταπτυχιακό}\\
\emph{Σχεδίαση της διεπαφής με το χρήστη και αξιολόγηση της ευχρηστίας
για εφαρμογές αμφίδρομης διαφήμισης στην ψηφιακή τηλεόραση, Επίβλεψη:
Γεώργιος Δουκίδης}\\
\emph{Μάρκετινγκ και Επικοινωνία, Οικονομικό Πανεπιστήμιο
Αθήνας}.\\[.2cm]
\note{1999}\textsc{Δίπλωμα}\\
\emph{Σύστημα για την ανάπτυξη εφαρμογών γεωγραφικής πληροφόρησης στον
παγκόσμιο ιστό, Επίβλεψη: Σταύρος Χριστοδουλάκης}\\
\emph{Ηλεκτρονικών Μηχανικών και Μηχανικών Ηλεκτρονικών
Υπολογιστών, Πολυτεχνείο Κρήτης}.\\[.2cm]

\section{Γλώσσες}
\emph{Ελληνικά} (Μητρική)\\
\emph{Αγγλικά} (Άριστο)\\
\emph{Ισπανικά} (Προχωρημένο)\\
\emph{Γαλλικά} (Βασικό)\\

\section{Διακρίσεις}
\noindent
\note{2003--2021}\textsc{Λίστα κορυφαίων επιστημόνων του Πανεπιστημίου
Στάνφορντ}\\
\emph{\href{https://doi.org/10.17632/btchxktzyw.5}{Βάση δεδομένων
επιστημονικών ετεροαναφορών}}\\
\note{2014--2016}\textsc{Επισκέπτης Αναπληρωτής Καθηγητής}\\
\emph{Norvegian University of Science and Technology (NTNU), Τροντχαϊμ,
Νορβηγία}\\
\note{2015}\textsc{Συν-υπεύθυνος προγράμματος (program co-chair)}\\
\emph{\href{https://icec2015.idi.ntnu.no/}{14th International Conference
on Entertainment Computing (ICEC) 2015}, NTNU, Norway}\\
\note{2012}\textsc{Προσκεκλημένος εισηγητής (keynote speaker)}\\
\emph{\href{http://sws2012.ime.usp.br/webmedia/keynotes.php}{WebMedia
2012 ACM SIGMM conference}, Sao Paulo, Brazil, 2012}\\
\note{2010--2011}\textsc{Υποτροφία Microsoft Research USA}\\
\emph{Υποτροφία για την εκτέλεση του ερευνητικού προγράμματος
Videopal}\\
\note{2009--2011}\textsc{Υποτροφία Marie Curie}\\
\emph{Υποτροφία για την εκτέλεση του ερευνητικού προγράμματος CULT
(Cultural Understanding through Learning and Technology)}\\
\note{2007}\textsc{Συν-υπεύθυνος προγράμματος (program co-chair)}\\
\emph{5\textsuperscript{ο} Ευρωπαϊκό συνεδρίο
\href{https://tvx.acm.org/2007/EuroITV-2007.html}{EuroITV 2007
Interactive TV as a Shared Experience} για την αμφίδρομη τηλεόραση, CWI,
Amsterdam, the Netherlands}\\
\note{2007}\textsc{Προσκεκλημένος εισηγητής (keynote speaker)}\\
\emph{Συνέδριο Eyes on Interactive Television (EyesonITV) 2007, Vasa,
Finland, May 2007.}\\
\note{2006}\textsc{Συν-υπεύθυνος προγράμματος (program co-chair)}\\
\emph{4\textsuperscript{ο} Ευρωπαϊκό συνεδρίο (EuroITV 2006 Beyond
Usability, Broadcast, and TV) για την αμφίδρομη τηλεόραση, ΟΠΑ, Αθήνα,
Ελλάδα.}\\
\note{2005}\textsc{Προσκεκλημένος εισηγητής (keynote speaker)}\\
\emph{3\textsuperscript{ο} ευρωπαϊκό συνέδριο European Conference on
Interactive Television (EuroITV) 2005 User Centred ITV Systems,
Programmes and Applications, Aalborg, Denmark.}\\
\note{2003}\textsc{Υποτροφία NSF (USA)}\\
\emph{National Science Foundation (NSF) για τη συμμετοχή (35\%
acceptance rate) στο πάνελ των διδακτορικών φοιτητών
(\href{http://www.chi2003.org/doctoral_consortium_program.html}{doctoral
consortium}) στο συνέδριο ACM SIGCHI conference on Human factors in
computing systems 2003, Ft Lauderdale, Florida, USA.}\\

\section{Έρευνα και τεχνολογική ανάπτυξη}
\noindent
\note{2023--}\textsc{Τεχνολογίες και πρακτικές συνεργασίας στο Github}\\
\emph{Πρόγραμμα εκπαίδευσης για την δια βίου μάθηση} ΚΕΔΙΒΙΜ\\[.2cm]
\note{2021--2022}\textsc{Κατασκευή Συστημάτων Διάδρασης}\\
\emph{Ανοιχτό βιβλίο για την 3βάθμια εκπαίδευση} ΚΑΛΛΙΠΟΣ+\\[.2cm]
\note{2020--2021}\textsc{Σχεδίαση Συνεργατικού Εκπαιδευτικού
Βιντεοπαιχνιδιού Ρόλων}\\
\emph{Επιστημονικά Υπεύθυνος} ΕΔΒΜ\\[.2cm]
\note{2016--2020}\textsc{Ανοιχτό περιεχόμενο και συμμετοχικό βιβλίο στην
εκπαίδευση}\\
\emph{Ανάπτυξη και συντήρηση συμπληρωματικού διαδραστικού και
συμμετοχικού περιεχομένου για \href{https://pibook.epidro.me}{βιβλίο για
την 3βάθμια εκπαίδευση}.} Αυτοχρηματοδότηση\\[.2cm]
\note{2015--2018}\textsc{TRAMOOC (Translation for Massive Open Online
Courses)}\\
\emph{Σχεδίαση και αξιολόγηση διεπαφής για την μετάφραση των υπότιτλων
σε βίντεο μαθήματα.} Ευρωπαϊκή Επιτροπή (Horizon2020)\\[.2cm]
\note{2012--2018}\textsc{Ανοιχτό λογισμικό στην έρευνα}\\
\emph{Ανάπτυξη και συντήρηση υπηρεσιών που βασίζονται σε έργα λογισμικού
ανοιχτού κώδικα, όπως \href{http://www.socialskip.org}{ανάλυση διάδρασης
με βίντεο}, \href{http://www.flutrack.org}{οπτικοποίηση δεδομένων από
κοινωνικά μέσα}, και τα \href{http://www.mapito.org}{γεωγραφικά συστήμα
πληροφόρησης}.} Αυτοχρηματοδότηση\\[.2cm]
\note{2010--2011}\textsc{VIDEOPAL}\\
\emph{Το έργο Videopal μελετά μια τεχνολογική παρέμβαση σε δύο
απομακρυσμένες αίθουσες διδασκαλίας, η μια βρίσκεται στην Ελλάδα και η
άλλη στις ΗΠΑ. Η τεχνολογική παρέμβαση εστιάζει στην ασύγχρονη
επικοινωνία μέσω βίντεο για παιδιά στις τελευταίες τάξεις του
δημοτικού.} Microsoft Research USA\\[.2cm]
\note{2009--2011}\textsc{CULT (Cultural Understanding through Learning
and Technology)}\\
\emph{Μελέτη των κοινωνικών επιπτώσεων της τοποθέτησης ενός συστήματος
διάχυτου υπολογισμού στον φυσικό χώρο. Το σύστημα περιλαμβάνει εύχρηστο
και ευχάριστο λογισμικό που επιτρέπει την κοινωνική διάδραση, τόσο από
μακρυά όσο και από κοντά, με πολλαπλά μέσα και συσκευές χρήστη. Τα
σχολεία που συμμετέχουν βρίσκονται τόσο στην Ελλάδα όσο και στο
εξωτερικό. Με αυτό τον τρόπο τα αποτελέσματα του έργου αφορούν
γενικότερα όλες εκείνες τις κοινότητες που βρίσκονται μακρυά από τις
πόλεις αλλά επιθυμούν να έχουν ίσες ευκαιρίες στην επικοινωνία και στην
συμμετοχή στα κοινά.} Ευρωπαϊκή Επιτροπή (Marie Curie FP7-ERG)\\[.2cm]
\note{2002--2010}\textsc{UITV (Understanding Interactive Television)}\\
\emph{Εκδότης του διεθνούς ηλεκτρονικού
\href{https://www.freelists.org/archive/uitv/}{ενημερωτικού δελτίου για
την αμφίδρομη ψηφιακή τηλεόραση} με περισσότερους από 800 εγγεγραμμένους
συνδρομητές (καθηγητές, ερευνητές, στελέχη διεθνών οπτικοακουστικών
οργανισμών και τηλεοπτικών σταθμών).} Αυτοχρηματοδότηση\\[.2cm]
\note{2006--2008}\textsc{MEDIACITY}\\
\emph{Μεταδιδακτορική έρευνα και διδασκαλία στην διεπιστημονική περιοχή
των \href{http://www.uni-weimar.de/mediaarchitecture}{νέων μέσων και της
αρχιτεκτονικής}.} Ευρωπαϊκή Επιτροπή (Marie Curie FP6-ΤοΚ)\\[.2cm]
\note{2004--2005}\textsc{TIRAMISU (The Innovative Rights and Access
Management Inter-platform Solution)}\\
\emph{Το έργο αυτό είχε αντικείμενο τη σχεδίαση και ανάπτυξη ενός
πληροφορικού συστήματος διαχείρισης δικαιωμάτων ψηφιακού περιεχομένου.
Απασχόληση με τη συλλογή των απαιτήσεων για τους χρήστες και τον
σχεδιασμό του επιχειρηματικού μοντέλου για την διανομή και τιμολόγηση
περιεχομένου πολυμέσων.} Ευρωπαϊκή Επιτροπή (FP6-IST)\\[.2cm]
\note{2001--2003}\textsc{CONTESSA (Content Transformation Engine
Supporting Universal Access)}\\
\emph{Στο έργο αυτό έγινε η ανάπτυξη εργαλείων συγγραφής πολυμεσικού
περιεχομένου, το οποίο θα μπορεί αυτόματα να προσαρμόζεται σε υβριδικά
δίκτυα (π.χ. μετάδοση μέσω Internet, TV, wireless) και σε πολλαπλές
καταναλωτικές συσκευές (π.χ. κινητό τηλέφωνο, υπολογιστής, τηλεόραση,
κτλ). Απασχόληση στη διαχείριση του έργου.} Ευρωπαϊκή Επιτροπή
(FP5-IST)\\[.2cm]
\note{2000--2001}\textsc{iMEDIA (Intelligent Mediation Environment for
Digital Interactive Advertising}\\
\emph{Στο έργο αυτό έγινε ανάπτυξη ενός πρωτότυπου συστήματος αμφίδρομης
διαφήμισης για την ψηφιακή τηλεόραση, το οποίο δημιουργεί ένα ξεχωριστό
διαφημιστικό διάλειμμα σε κάθε οικιακό αποκωδικοποιητή, με βάση τα
προφίλ των χρηστών. Απασχόληση στη σχεδίαση και αξιολόγηση της διεπαφής
με το χρήστη.} Ευρωπαϊκή Επιτροπή (FP5-IST)\\[.2cm]
\note{1997--1999}\textsc{CAMPIELLO (Interacting in collaborative
environments to promote and sustain the meeting between inhabitants and
tourists}\\
\emph{Στο έργο αυτό μελετήθηκαν οι δυνατότητες για χρήση κοινωνικών
συστημάτων γεωγραφικής πληροφόρησης σε περιοχές με πολιτιστικό
ενδιαφέρον και έντονη τουριστική ανάπτυξη, όπως είναι τα Χανιά και η
Βενετία. Σχεδίαση και υλοποίηση της διεπαφής με το χρήστη για ένα
γεωγραφικό σύστημα πληροφόρησης μέσω Web.} Ευρωπαϊκή Επιτροπή
(FP4-ESPRIT)\\[.2cm]

\section{Διδασκαλία}

\subsection{Επίβλεψη διδακτορικών διατριβών}
\noindent
\note{2018--}\textsc{Κωνσταντίνος Πατηνιώτης}\\
\emph{Βίντεο-παιχνίδια και Εκπαίδευση}\\[.2cm]
\note{2017--}\textsc{Μανούσος Καμηλάκης}\\
\emph{Γεωγραφικά Συστήματα Πληροφόρησης}\\[.2cm]
\note{2016--2021}\textsc{Διογένης Αλεξανδράκης}\\
\emph{Κοινωνικά και Συνεργατικά Συστήματα}\\[.2cm]
\note{2014--2019}\textsc{Αλέξανδρος Μερκούρης}\\
\emph{Ρομποτική και Εκπαίδευση}\\[.2cm]
\note{2013--2017}\textsc{Βαρβάρα Γαρνέλη}\\
\emph{Βίντεο-παιχνίδια και Εκπαίδευση}\\[.2cm]

\subsection{Επίβλεψη διπλωματικών και πτυχιακών εργασιών}
\noindent
\note{2022}\textsc{Κωνσταντίνος Αστροπεκάκης, ΕΑΠ}\\
\emph{Συνεργατικά συστήματα επαυξημένης ψηφιακής εκπαίδευσης}\\[.2cm]
\note{2022}\textsc{Νικόλαος Δουλαβέρας, ΕΑΠ}\\
\emph{Ανάλυσης συμπεριφοράς πλήθους σε δημόσιο χώρο}\\[.2cm]
\note{2020}\textsc{Ευαγγελία Μπατόγλου, ΕΑΠ}\\
\emph{Συμπληρωματικές διεπαφές επεξεργασίας γεωγραφικού χάρτη}\\[.2cm]
\note{2019}\textsc{Μπέτυ Χωριανοπούλου, ΕΑΠ}\\
\emph{Διδακτική της ρομποτικής στην πρωτοβάθμια εκπαίδευση}\\[.2cm]
\note{2018}\textsc{Ζήσης Δημητριάδης, ΕΑΠ}\\
\emph{Επαυξημένο φυσικό βιβλίο}\\[.2cm]
\note{2017}\textsc{Αντώνης Παπαπολύζος, ΕΑΠ}\\
\emph{Κινητός υπολογισμός και γεωγραφικοί χάρτες}\\[.2cm]
\note{2016}\textsc{Σήφης Σημιανάκης, ΕΑΠ}\\
\emph{Κινητός υπολογισμός και βίντεο}\\[.2cm]
\note{2015}\textsc{Ιπποκράτης Καπενεκάκης, ΕΑΠ}\\
\emph{Κινητός υπολογισμός και γεωγραφικοί χάρτες}\\[.2cm]
\note{2015}\textsc{Κωνσταντίνος Παρδάλης, ΙΠ}\\
\emph{Πλατφόρμα διάχειρισης ψηφιακών γεωγραφικών χαρτών}\\[.2cm]
\note{2014}\textsc{Αλέξανδρος Μερκούρης, ΕΑΠ}\\
\emph{Ρομποτική και εκπαίδευση}\\[.2cm]
\note{2013}\textsc{Κάρολος Ταλβής, ΙΠ}\\
\emph{Οπτικοποίηση δεδομένων από κοινωνικά δίκτυα}\\[.2cm]
\note{2013}\textsc{Βαρβάρα Γαρνέλη, ΙΠ}\\
\emph{Βίντεο-παιχνίδια και εκπαίδευση}\\[.2cm]
\note{2012}\textsc{Χριστίνα Ιλιούδη, ΙΠ}\\
\emph{Βίντεο και εκπαίδευση}\\[.2cm]
\note{2011}\textsc{Ιωάννης Λευτεριώτης, ΙΠ}\\
\emph{Πολυαπτική διάδραση}\\[.2cm]
\note{2011}\textsc{Χρυσούλα Γκονέλα, ΙΠ}\\
\emph{Διάδραση και βίντεο}\\[.2cm]
\note{2010}\textsc{Γιούλη Σταματούκου, ΕΑΠ}\\
\emph{Εκπαιδευτικό βίντεο-παιχνίδι}\\[.2cm]
\note{2010}\textsc{Κωνσταντίνος Μικάλεφ, ΕΑΠ}\\
\emph{Εκπαίδευση και μουσείο}\\[.2cm]
\note{2007}\textsc{Χαράλαμπος Κουτσουρελάκης, ΕΑΠ}\\
\emph{Ευχρηστία κινητών τηλεφώνων}\\[.2cm]

\subsection{Διδασκαλία μεταπτυχιακών μαθημάτων}
\noindent
\note{2011--}\textsc{Σχεδίαση και Ανάλυση Λογισμικού-Υλικού (ΣΔΥ60),
ΕΑΠ}\\
\emph{ΣΕΠ, Συντονιστής}\\[.2cm]
\note{2020--}\textsc{Οπτικοποίηση της Πληροφορίας, ΠΠ}\\
\emph{Εργασίες}\\[.2cm]
\note{2009--2013, 2019--}\textsc{Κοινωνικά και Συνεργατικά Συστήματα
(Συνεργατικές Εφαρμογές), ΙΠ}\\
\emph{Εργασίες}\\[.2cm]
\note{2019--2020}\textsc{Διάχυτος Υπολογισμός, ΠΠ}\\
\emph{Εργασίες, 50\%}\\[.2cm]
\note{2019--2020}\textsc{Συνεργατικές Τεχνολογίες, ΠΠ}\\
\emph{Εργασίες, 50\%}\\[.2cm]
\note{2009--2013}\textsc{Σχεδίαση της Διάδρασης Ανθρώπου-Υπολογιστή,
ΙΠ}\\
\emph{Διαλέξεις}\\[.2cm]
\note{2010--2011}\textsc{Εξειδικεύσεις στην Τεχνολογία Λογισμικού
(ΠΛΣ60), ΕΑΠ}\\
\emph{ΣΕΠ}\\[.2cm]
\note{2009--2010}\textsc{Σχεδιασμός και Διαχείριση Λογισμικού (ΠΛΣ61),
ΕΑΠ}\\
\emph{ΣΕΠ}\\[.2cm]
\note{2007--2008}\textsc{Διαδραστική τηλεόραση, Πανεπιστήμιο Τεχνών
Βερολίνου}\\
\emph{Διαλέξεις, Εργασίες}\\[.2cm]
\note{2006--2008}\textsc{Διαδραστική σχεδίαση, Πανεπιστήμιο Αιγαίου}\\
\emph{Διαλέξεις}\\[.2cm]

\subsection{Διδασκαλία προπτυχιακών μαθημάτων}
\noindent
\note{2020--}\textsc{Οπτικοποίηση της Πληροφορίας, ΙΠ}\\
\emph{Εργασίες}\\[.2cm]
\note{2008--}\textsc{Επικοινωνία Ανθρώπου-Υπολογιστή, ΙΠ}\\
\emph{Διαλέξεις, Εργαστήριο}\\[.2cm]
\note{2008--}\textsc{Τεχνολογία Λογισμικού, ΙΠ}\\
\emph{Διαλέξεις, Εργαστήριο}\\[.2cm]
\note{2011--}\textsc{Κοινωνικά και Συνεργατικά Συστήματα (Κινητά και
Κοινωνικά Μέσα), ΙΠ}\\
\emph{Διαλέξεις, Εργαστήριο}\\[.2cm]
\note{2009--2019}\textsc{Πολυμέσα, ΙΠ}\\
\emph{Διαλέξεις, Εργαστήριο}\\[.2cm]
\note{2011--2014}\textsc{Απανταχού υπολογίζειν, ΙΠ}\\
\emph{Εργασίες}\\[.2cm]
\note{2008--2009}\textsc{ΤΠΕ στον Τουρισμό, ΙΠ}\\
\emph{Διαλέξεις, Εργαστήριο}\\[.2cm]
\note{2007--2008}\textsc{Τέχνη Υπολογιστή και Τέχνη διαδικτύου, ΙΠ}\\
\emph{Διαλέξεις, Εργαστήριο}\\[.2cm]
\note{2006--2008}\textsc{Πολυμέσα, ΠΑ}\\
\emph{Διαλέξεις, 50\%}\\[.2cm]
\note{2006--2007}\textsc{Προγραμματισμός, ΠΑ}\\
\emph{Εργαστήριο, 50\%}\\[.2cm]
\note{2006--2008}\textsc{Οπτικοακουστικές τεχνικές, ΠΑ}\\
\emph{Εργαστήριο}\\[.2cm]
\note{2006--2008}\textsc{Θεωρία της επικοινωνίας, ΠΑ}\\
\emph{Διαλέξεις}\\[.2cm]
\note{2004--2005}\textsc{Ψηφιακά μέσα, ΟΠΑ}\\
\emph{Διαλέξεις, 50\%}\\[.2cm]

\subsection{Προσκεκλημένος ομιλητής}
\noindent
\note{2022}\textsc{Ευχρηστία και Ποιότητα Διάδρασης}\\
\emph{Προσκεκλημένος ομιλητής, Διεθνές Συνέδριο, Ελληνική Ομάδα
Ευχρηστίας, ΕΚΠΑ}, Αθήνα, Ελλάδα\\[.2cm]
\note{2021}\textsc{Προσομοίωση και Εξομοίωση στο Λογισμικό Δικτυακής
Συσνεργασίας}\\
\emph{Προσκεκλημένος ομιλητής, Επέτειος 20 ετών, ΟΠΑ}, Αθήνα,
Ελλάδα\\[.2cm]
\note{2015}\textsc{Προγραμματισμός της διάδρασης και καινοτομία}\\
\emph{Προσκεκλημένος ομιλητής, Ευχρηστία και Καινοτομία, ΕΚΠΑ}, Αθήνα,
Ελλάδα\\[.2cm]
\note{2014}\textsc{Open system and open data for epidemiology}\\
\emph{Προσκεκλημένος ομιλητής, Madeira Interactive Technologies
Institute}, Μαδέϊρα, Πορτογαλία\\[.2cm]
\note{2013}\textsc{Open system and open data for epidemiology}\\
\emph{Προσκεκλημένος ομιλητής, Health Informatics Day, ΕΚΠΑ}, Αθήνα,
Ελλάδα\\[.2cm]
\note{2012}\textsc{Ubiquitous computing and interaction}\\
\emph{Σεμινάριο, Aarhus University}, Aarhus, Denmark\\[.2cm]
\note{2012}\textsc{Interaction design for ubiquitous learning and
entertainment through video technologies}\\
\emph{Σεμινάριο, Sao Paulo University}, Sao Carlos, Brazil\\[.2cm]
\note{2012}\textsc{Graphical interfaces for Navigating Web video}\\
\emph{Προσκεκλημένος ομιλητής, Chalmers University of
Technology}, Gothenburg, Sweden\\[.2cm]
\note{2012}\textsc{Digital media pedagogy}\\
\emph{Προσκεκλημένος ομιλητής, KTH}, Stockholm, Sweden\\[.2cm]
\note{2012}\textsc{User-centered design and development methods for TV
and video interaction}\\
\emph{Προσκεκλημένος ομιλητής, University of Aveiro}, Aveiro,
Portugal\\[.2cm]
\note{2010}\textsc{Η διδασκαλία και η έρευνα στην περιοχή της
Επικοινωνίας Ανθρώπου-Υπολογιστή στο Ιόνιο Πανεπιστήμιο}\\
\emph{Σεμινάριο, Ημέρες Ευχρηστίας και Προσβασιμότητας 2010,
ΕΚΠΑ}, Αθήνα, Ελλάδα\\[.2cm]
\note{2007}\textsc{Από τα μέσα μαζικής επικοινωνίας στα διαδραστικά
πολυμέσα}\\
\emph{Προσκεκλημένος ομιλητής. Ιόνιο Πανεπιστήμιο}, Κέρκυρα,
Ελλάδα\\[.2cm]
\note{2006}\textsc{Interaction Design for Ambient Systems}\\
\emph{Προσκεκλημένος ομιλητής. Αρχιτεκτονικά μέσα επικοινωνίας.
Πανεπιστήμιο Κύπρου}, Λευκωσία, Κύπρος\\[.2cm]
\note{2006}\textsc{Supporting the Social Uses of Interactive TV}\\
\emph{Προσκεκλημένος ομιλητής, Social computing course, University of
the Arts (UDK)}, Βερολίνο, Γερμανία\\[.2cm]
\note{2005}\textsc{User Interface Design for Interactive TV}\\
\emph{Προσκεκλημένος ομιλητής, HCI course, University College
London}, London, England\\[.2cm]
\note{2002}\textsc{Σχεδίαση της αλληλεπίδρασης με το χρήστη για
εφαρμογές αμφίδρομης τηλεόρασης}\\
\emph{Διεθνές μεταπτυχιακό GEM, Οικονομικό Πανεπιστήμιο Αθήνας}, Αθήνα,
Ελλάδα\\[.2cm]

\hypertarget{ux3c5ux3c0ux3ccux3bcux3bdux3b7ux3bcux3b1-ux3b3ux3b9ux3b1-ux3c4ux3b7ux3bd-ux3b4ux3b9ux3b4ux3b1ux3baux3c4ux3b9ux3baux3ae-ux3c4ux3c9ux3bd-ux3bcux3b1ux3b8ux3b7ux3bcux3acux3c4ux3c9ux3bd}{%
\section{Υπόμνημα για την διδακτική των
μαθημάτων}\label{ux3c5ux3c0ux3ccux3bcux3bdux3b7ux3bcux3b1-ux3b3ux3b9ux3b1-ux3c4ux3b7ux3bd-ux3b4ux3b9ux3b4ux3b1ux3baux3c4ux3b9ux3baux3ae-ux3c4ux3c9ux3bd-ux3bcux3b1ux3b8ux3b7ux3bcux3acux3c4ux3c9ux3bd}}

Η βασική μαθησιακή φιλοσοφία που διατρέχει όλα τα μθήματα είναι ότι οι
φοιτητές θα πρέπει να έχουν μεγάλη συμμετοχή στο περιεχόμενο του
μαθήματος κυρίως μέσω των εργασιών, έτσι ώστε να αποκτήσουν γνώσεις με
ενεργετικό τρόπο και να αποκτήσουν χρήσιμες δεξιότητες. Για να αυξήσουμε
την συμμετοχή των φοιτητών βασιζόμαστε στους παρακάτω πυλώνες:

\begin{itemize}
\tightlist
\item
  Ιδιοκτησία της εργασίας με ατομικό ορισμό των προδιαγραφών
\item
  Κοινό αποθετήριο με τα παραδοτέα εργασιών για όλους
\item
  Δύο τουλάχιστον ενδιάμεσα παραδοτέα με προφορική παρουσίαση
\end{itemize}

Τα μαθήματα δεν βασίζονται στις διαλέξεις, αλλά στις ομότιμες
συζητήσεις. Οι σποραδικές και ευκαιριακές διαλέξεις του μαθήματος
χρησιμοποιούνται για να εμπνεύσουν τους φοιτητές μέσω των πρακτικών
εφαρμογών που έχει η αντίστοιχη γνώση. Το πρώτο παραδοτέο καλεί τους
φοιτητές να διαλέξουν μια από τις θεματικές εργασίες(π.χ., κατασκευή
εκπαιδευτικού βιντεοπαιχνιδιού) και να ορίσουν μόνοι τους τις
προδιαγραφές (π.χ., μαθηματικά 6ης δημοτικού) μέσα στο πλαίσιο που
περιγράφει η αντίστοιχη εκφώνηση. Με αυτόν τον τρόπο υπάρχουν
συγκεκριμένα και κοινά κριτήρια βαθμολόγησης, αλλά και περιθώριο ο κάθε
φοιτητής να κάνει αυτό που τον ενδιαφέρει. Ολοι οι φοιτητές παραδίδουν
στην ίδια περιοχή, έτσι ώστε να υπάρχει διαφάνεια και δυνατότητα να
μαθαίνουν από τα λάθη/σωστά των άλλων. Ταυτόχρονα τα δύο τουλάχιστον
ενδιάμεσα παραδοτέα παρουσιάζονται προφορικά, έτσι ώστε να υπάρχει
διαπροσωπική επικοινωνία και συμβουλές για βελτίωση. Καθώς το μάθημα
προχωράει, τα έργα των φοιτητών δίνουν στον διδάσκοντα καλύτερη εικόνα
της πρόδοου και επιτρέπουν διορθωτικές κινήσεις. Επιπλέον, οι διαλέξεις
αρχίζουν να περιστρέφονται λιγότερο γύρω από την ``ύλη'' και περισσότερο
πάνω στα έργα των φοιτητών (π.χ., επιτυχημένα έργα, σημεία για
βελτίωση), έτσι ώστε να υπάρχει μια αίσθηση επίδρασης. Η παραπάνω
διαδικασία είναι οργανωμένη σε αποθετήρια κώδικα στο σύστημα github:
https://github.com/courses-ionio Για την βέλτιστη εφαρμογή των παραπάνω
μεθόδων απαιτείται και η συμμετοχή των βοηθών του μαθήματος (υποψήφιοι
διδάκτορες) με αναλογία 20 περίπου φοιτητές για κάθε βοηθό.

\hypertarget{ux3b5ux3c0ux3b9ux3c3ux3c4ux3b7ux3bcux3bfux3bdux3b9ux3baux3ae-ux3baux3b1ux3b9-ux3b4ux3b9ux3bfux3b9ux3baux3b7ux3c4ux3b9ux3baux3ae-ux3c5ux3c0ux3b7ux3c1ux3b5ux3c3ux3afux3b1}{%
\section{Επιστημονική και διοικητική
υπηρεσία}\label{ux3b5ux3c0ux3b9ux3c3ux3c4ux3b7ux3bcux3bfux3bdux3b9ux3baux3ae-ux3baux3b1ux3b9-ux3b4ux3b9ux3bfux3b9ux3baux3b7ux3c4ux3b9ux3baux3ae-ux3c5ux3c0ux3b7ux3c1ux3b5ux3c3ux3afux3b1}}

\textbf{Συντακτική επιτροπή:} ACM Computers in Entertainment, Elsevier
Entertainment Computing, Journal of Virtual Reality and Broadcasting

\textbf{Επιτροπές συνεδρίων:} 14th International Conference on
Entertainment Computing (ICEC 2015), 13th International Conference on
Entertainment Computing (ICEC 2014), ACM TVX 2014, EuroITV 2005-2013,
21st ACM International Conference on Multimedia (ACM Multimedia 2013),
Workshop on Multi-User Services for Social TV (MUSST2013), CVRB 2013 -
1st International Conference on Virtual Reality and Broadcasting,
International Conference on MultiMedia Modeling (MMM 2012, 2014-2015),
4th International Conference on Embedded and Multimedia Computing
(EM-Com 2009), 6\textsuperscript{th} International Conference on
Ubiquitous Intelligence and Computing (UIC 2009), 1\textsuperscript{st}
International Conference on Designing Interactive User Experiences for
TV and Video (UXTV08), 2nd International Conference on Multimedia and
Ubiquitous Engineering (MUE 2008), IFIP Entertainment Computing
Symposium (ECS-2008), Ambisys 2008 Workshop on Ambient Media Delivery
and Interactive Television (AMDIT 2008), European Interactive TV
(EuroITV) conference (2005-2010), IET Intelligent Environments 2007, IET
Intelligent Environments 2008, ACM SIGCHI 2006 Workshop on sociable and
mobile ITV, ACM SIGCHI 2007 workshop on Shared Encounters: Content
Sharing as Social Glue in Everyday Places, ACM SIGCHI 2007 Workshop on
Supporting non-professional users in the new media landscape

\textbf{Ομάδες εργασίας:} Vice-Chair of the
\href{http://www.ifip.org/bulletin/bulltcs/memtc14.htm\#WG146}{IFIP
Working Group on Interactive TV}, Technical Committee on Entertainment
Computing (TC14), International Federation for Information Processing
(IFIP)

\textbf{Μέλος:} IFIP, TEE, ACM

\textbf{Κριτής:}

\emph{Περιοδικά:} Computers and Education (Elsevier), The International
Review of Research in Open and Distance Learning, Journal of Computer
Assisted Learning (Wiley), Human Technology: An Interdisciplinary
Journal on Humans in ICT Environments, Journal of Experimental Child
Psychology (Elsevier), Entertainment Computing (Elsevier), Multimedia
Systems Journal (ACM/Springer), Information Sciences Journal (Elsevier),
Personal and Ubiquitous Computing (Springer), Transactions on Multimedia
(IEEE), Transactions on Multimedia Computing, Communications and
Applications (ACM TOMCCAP), Multimedia Tools and Applications
(Springer), International Journal of Human-Computer Interaction (Taylor
and Francis), Behaviour and Information Technology (Taylor and Francis),
Telematics and Informatics (Elsevier), Computers in Entertainment (ACM),
Journal of Virtual Reality and Broadcasting, International Journal of
E-Services and Mobile Applications (IGI)

\emph{Συνέδρια:} UbiComp 2012, 18th International Conference on
MultiMedia Modeling (MMM 2012), 7th IEEE International Workshop on
Networking Issues in Multimedia Entertainment (NIME\textquotesingle11),
7th International Conference on Pervasive Computing (Pervasive 2009),
ACM SIGCHI Papers and Notes (2007-2013), ACM SIGCHI UIST Demos (2008),
ACM SIGCHI UIST Papers (2007), IEΤ Intelligent Environments (2007-2008),
ACM SIGCHI Student Papers (2003-2006), ACM SIGCHI Research in Progress
(2003-2006, 2008, 2011), ACM SAC 2004 - Special Track on Ubiquitous
Computing, Cross Media Service Delivery Conference 2003

\textbf{Ανάπτυξη συστήματος:} Υλοποίηση συστήματος διαχείρισης και
πλοήγησης περιεχομένου για την πύλη UITV.INFO
\href{http://uitv.epidro.me}{\uline{http://uitv.epidro.me}} Το σύστημα
επιτρέπει την εύκολη ενημέρωση της πήλης με περιεχόμενο σχετικά με την
έρευνα στην ψηφιακή τηλεόραση. Αποθετήριο ανοικτού κώδικα για το
εκπαιδευτικό και ερευνητικο λογισμικό στο τμήμα:
\href{https://github.com/ioniodi}{\uline{https://github.com/ioniodi}}

\textbf{Σύμβουλος έκδοσης:} Συνεργασία (2006-2008) με την φοιτητική
ομάδα που είναι υπεύθυνη για την διαδικτυακή πύλη του Πανεπιστημίου
Αιγαίου (\href{http://my.aegean.gr/}{\uline{http://my.aegean.gr}}).

\textbf{Διοικητικό έργο και επιτροπές (Τμήμα Πληροφορικής, Ιόνιο
Πανεπιστήμιο):} Προγράμματος σπουδών, πρακτικής άσκησης, αναμόρφωσης της
κατεύθυνσης ανθρωπιστικές και κοινωνικές επιστήμες,
\href{https://github.com/ioniodi/sitegr}{συνεργατικής ιστοσελίδας
τμήματος}, οδηγού σπουδών, μετασχηματισμός του github σε πλατφόρμα
συνεργατικής μάθησης, εξ αποστάσεως διδασκαλία και αξιολόγηση.

\textbf{Εκλεκτορικά}: Διαδραστική Σχεδίαση (Πανεπιστήμιο Αιγαίου,
μέλος), Ψηφιακά Μέσα (Ιόνιο Πανεπιστήμιο, μέλος), Εκπαιδευτική
Τεχνολογία (Ιόνιο Πανεπιστήμιο, μέλος), Γλώσσες Προγραμματισμού
(Χαροκόπειο Πανεπιστήμιο, εισηγητική)

\hypertarget{ux3c5ux3c0ux3ccux3bcux3bdux3b7ux3bcux3b1-ux3b3ux3b9ux3b1-ux3c4ux3b1-ux3b5ux3c0ux3b9ux3c3ux3c4ux3b7ux3bcux3bfux3bdux3b9ux3baux3ac-ux3b4ux3b7ux3bcux3bfux3c3ux3b9ux3b5ux3cdux3bcux3b1ux3c4ux3b1}{%
\section{Υπόμνημα για τα επιστημονικά
δημοσιεύματα}\label{ux3c5ux3c0ux3ccux3bcux3bdux3b7ux3bcux3b1-ux3b3ux3b9ux3b1-ux3c4ux3b1-ux3b5ux3c0ux3b9ux3c3ux3c4ux3b7ux3bcux3bfux3bdux3b9ux3baux3ac-ux3b4ux3b7ux3bcux3bfux3c3ux3b9ux3b5ux3cdux3bcux3b1ux3c4ux3b1}}

Το μεγαλύτερο μέρος της έρευνας αφορά την κατασκευή λογισμικού για την
διάδραση ανθρώπου-υπολογιστή. Ταυτόχρονα, η μεθοδολογία αντιμετώπισης
των ερευνητικών θεμάτων διατρέχει ένα διεπιστημονικό άξονα, με βασικούς
πυλώνες:

\begin{itemize}
\item
  Ανάπτυξη εργαλείων λογισμικού,
\item
  Ανθρωποκεντρική σχεδίαση της αλληλεπίδρασης με τον χρήστη,
\item
  Πειράματα με χρήστες στο εργαστήριο ή τις μελέτες στο πεδίο
\end{itemize}

Τα ερευνητικά ενδιαφέροντα εστιάζουν κυρίως στις παρακάτω περιοχές:

\emph{Επικοινωνία ανθρώπου-υπολογιστή:} Το βασικό υπόβαθρο καθώς και ενα
σημαντικό τμήμα της έρευνας, όπως φαίνεται και από την πτυχιακή εργασία
(Τ1), την μεταπτυχιακή διατριβή (Τ2) και τη διδακτορική διατριβή (Τ3),
αφορά κυρίως την σχεδίαση λογισμικού διάδρασης ανθρώπου-υπολογιστή. Το
αντικείμενο αυτό αντιμετωπίζεται από διαφορετικές πλευρές και σε
διαφορετικά πεδία εφαρμογής: 1) Στην πτυχιακή εργασία (Τ1) έγινε
υλοποίηση της αλληλεπίδρασης με τον χρήστη για ένα σύστημα γεωγραφικής
πληροφόρησης με ηλεκτρονικό χάρτη μέσω Web, 2) στην μεταπτυχιακή
διατριβή (Τ2) έγινε μελέτη και αξιολόγηση της αλληλεπίδρασης με τον
χρήστη για ένα σύστημα αμφίδρομης διαφήμισης στην ψηφιακή τηλεόραση, και
3) στην διδακτορική διατριβή (Τ3) έγινε σχεδίαση, υλοποίηση και
αξιολόγηση της αλληλεπίδρασης του χρήστη με εφαρμογές αμφίδρομης
τηλεόρασης και ειδικά με ένα υπόδειγμα μουσικής τηλεόρασης. Εκτός από
την περίπτωση της τηλεόρασης (J1, J2, J3, J4, J7), επιπλέον μελέτες της
διάδρασης με τον χρήστη έχουν γίνει για την περίπτωση των έξυπνων
κινητών τηλεφώνων (J9, J10, J14), των γεωγραφικών συστημάτων
πληροφόρησης (J16, J24), των συστημάτων διαπροσωπικής επικοινωνίας (J12,
J13), των μεγάλων πολυαπτικών οθονών (P23, P24, P33, J22), και της
εκπαιδευτικής ρομποτικής (J25).

\emph{Διαδραστική τηλεόραση και βίντεο:} Η βασική φιλοσοφία που
διατρέχει την έρευνα για την ψηφιακή τηλεόραση είναι η θεώρηση των
χρηστών ως θεατές ή ως παραγωγούς οπτικοακουστικού περιεχομένου (J1, J3,
J4, J7, B1). Η προσέγγιση αυτή έρχεται σε αντίθεση με την ισχύουσα
θεώρηση του χρήστη αλληλεπιδραστικών πολυμέσων ως χρήστη επιτραπέζιου
υπολογιστή. Εχει γίνει εκτεταμένη εμπειρική έρευνα σε θέματα όπως: 1)
Σχεδιασμός και ανάπτυξη εργαλείων προγραμματισμού που είναι κατάλληλα
για τη ροή εργασίας των παραγωγών τηλεοπτικού περιεχομένου (J1, J4), 2)
σχεδίαση της διεπαφής έτσι ώστε να διευκολύνει την αλληλεπίδραση του
τηλεθεατή με την πλούσια οπτική γλώσσα του τηλεοπτικού προγράμματος (J2,
J3, J5, P2, P4). Ακόμη, έγινε σχεδιασμός ενός μοντέλου που διευκολύνει
την καθολική πρόσβαση των τηλεθεατών σε αμφίδρομο τηλεοπτικό περιεχόμενο
(P5). 3) Ειδική μεθοδολογία αξιολόγησης της ευχρηστίας (J4), η οποία
λαμβάνει υπόψιν την χρήση και τις ανάγκες του τηλεοπτικού κοινού (P7).
Καθώς το βίντεο έγινε μετά το 2005 σταδιακά διαθέσιμο και μέσω της
ευρυζωνικότητας, η έρευνα αυτή έχει επεκταθεί και συνεχίζεται στην
μελέτη της διάδρασης με δικτυακό βίντεο (J11, J15, J18, J19, P20).

\emph{Εκπαιδευτικές τεχνολογίες και διδακτική του προγραμματισμού:} Η
χρήση των υπολογιστών για την διευκόλυνση της μάθησης και ειδικά η
διδακτική του προγραμματισμού έχουν σημαντικό ρόλο στην εκπαίδευση
γενικά και ειδικά σε αυτήν της πληροφορικής. Η έρευνα για την
διευκόλυνση της μάθησης εστιάζει στις τεχνολογίες του βίντεο και της
διάδρασης (J20, J27), καθώς και στις τεχνολογίες των εκπαιδευτικών
βιντεοπαιχνιδιών (J17, J23, J28). Η τρέχουσα έρευνα για την διδακτική
του προγραμματισμού, εστιάζει στην χρήση της ρομποτικής (J25).

\emph{Κινητός και διάχυτος υπολογισμός:} Ενα σημαντικό και αυξανόμενο
τμήμα της έρευνας αφορά τη σχεδίαση και τις κοινωνικές πτυχές των
τεχνολογιών κινητού και διάχυτου υπολογισμού. Η έρευνα σε αυτές τις
περιοχές μπορεί να αναδείξει περισσότερο σημαντικές πτυχές των
υπολογιστικών και δικτυακών εφαρμογών σε σχέση με τις παραδοσιακές που
έχουν έμφαση κυρίως στον χώρο της εργασίας. Εκτός από τα συστήμα
ψυχαγωγίας μέσω της τηλεόρασης (J1, J2), έχει γίνει μελέτη σε κινητά και
διάχυτα συστήματα (J14, J26). Ακόμη, έγινε κατασκευή λογισμικού
διάδρασης για συνεργασία πολλών χρηστών πάνω στην ίδια μεγάλη πολυαπτική
οθόνη (J22). Τέλος, γίνεται μελέτη στα γεωγραφικά συστήματα πληροφορήσης
(J16, J24) και στις κινητές τεχνολογίες θέσης αφού έχουν κεντρικό ρόλο
στην καθημερινότητα του χρήστη.

\emph{Συνεργατικά και κοινωνικά μέσα:} Σε ένα δεύτερο επίπεδο, επιπλέον
έρευνα έχει γίνει στο πεδίο των κοινωνικών μέσων, η οποία ενισχύει και
διευρύνει το πεδίο εφαρμογής των καθιερωμένων επιχειρηματικών μοντέλων
της βιομηχανίας της τηλεόρασης (J6) σε νέους τομείς, όπως το διαδίκτυο
και τα κινητά τηλέφωνα (J5). Επίσης, έχει γίνει μελέτη των προτιμήσεων
και της συμπεριφοράς των χρηστών σε σε εικονικούς κόσμους (J8). Τέλος
γίνεται ανάπτυξη συστήματος και ανάλυσης των δεδομένων από κοινωνικά
δίκτυα για την διευκόλυνση στον εντοπισμό κοινωνικών φαινομένων όπως η
εξάπλωση της γρίπης (J21).

%bibliography
\section{Βιβλιογραφία}

\bibsection{Διατριβές}{../_bib/thesis.bib}{Τ}

\end{greek}

\begin{english}

\bibsection{Βιβλία}{../_bib/books.bib}{B}

\bibsection{Επιμέλεια βιβλίων}{../_bib/editor.bib}{E}

\bibsection{Επιμέλεια περιοδικών}{../_bib/guest.bib}{G}

\bibsection{Περιοδικά}{../_bib/journals.bib}{J}

\bibsection{Συνέδρια}{../_bib/conferences.bib}{P}

\bibsection{Συμπόσια}{../_bib/workshops.bib}{W}

\bibsection{Κεφάλαια σε βιβλία}{../_bib/chapters.bib}{C}

\end{english}


%\vspace{1cm}
\vfill{}
%\hrulefill

\begin{center}
{\scriptsize  Τελευταία ενημέρωση: \today\- •\-
% ---- PLEASE LEAVE THIS BACKLINK FOR ATTRIBUTION AS PER CC-LICENSE
Η σελιδοποίση έγινε με το \href{https://github.com/mrzool/cv-boilerplate}{
\fontspec{Times New Roman}\XeTeX }\\
% ---- FILL IN THE FULL URL TO YOUR CV HERE
\href{http://users.ionio.gr/choko}{users.ionio.gr/choko}}
\end{center}

\end{document}
